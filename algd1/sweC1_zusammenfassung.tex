\documentclass[11pt,a4paper,ngerman]{article}
\usepackage[T1]{fontenc}
\usepackage{textcomp}
\usepackage{babel}
\date{16. September 2013}
\author{Fabian~Kuenzli}
\title{Zusammenfassung - Software Engineering 1}
\begin{document}
\maketitle
\newpage
\tableofcontents
\newpage
\section{Woche 1}
\subsection{Was ist Software?}

Zwei Zitate:
\begin{quote}
''Software is arguably the world's most important industry. The presence of software has made possible many new businesses and is responsible for increased efficiencies in most traditional businesses.''

\textit{Grady Booch (Communications of the ACM, M\"atz 2001)}

\end{quote}

\noindent Bei ''Software'' handelt es sich also nicht nur um das reine Produkt. Durch Software ist auch eine sehr grosse, (f�r die Schweiz) wichtige Industrie entstanden.

\begin{quote}
''Die Gesamtheit von Softwarekomponenten, die als Ganzes entwickelt, vertrieben, angewendet und gewartet werden.''

\textit{R. Dumme, Software Engineering, 2000}

\end{quote}

\noindent \textbf{Bestandteile von Software sind}
\begin{itemize}
\item Computer-Programm(e)
\item Installationsprogramm(e)
\item Benutzerhandbuch
\item Dokumentation
\end{itemize}

\noindent \textbf{Merkmale von Software}
\begin{itemize}
\item Software ist immateriell - die Qualit�t ist schwer zu beurteilen
\item Software wird nicht durch physikalische Gesetze begrenzt
\item Produkt der menschlichen Kreativit�t mit Gestaltungsfreiheit
\item Software ist leicht und schnell �nderbar (nicht nur positiv)
\item Software wird meist von Menschen bedient
\item Software hat eine l�ngere Lebenszeit als Hardware
\item Software unterliegt keinem Verschleiss
\item Software altert, da sich die Umgebung ver�ndert
\item Software wird zunehmend komplexer
\end{itemize}
\subsection{Was ist Software Engineering?}
\subsubsection{Ursprung}
In den Urspr\"ungen der Softwareentwicklung handelte es sich bei ''Software'' um sehr kleine Werkzeuge mit geringer Komplexit�t. Ein Grossteil der Projektkosten floss in die Hardware, Software kostete zu diesem Zeitpunkt wenig.\newline
Bald wurden die M�glichkeiten von Software erkannt und die Kosten f�r Software �berstiegen die Kosten der Hardware. Es gab immer mehr gescheiterte Softwareprojekte, was Mitte der der Sechzigerjahre zur Software-Krise f\"uhrte. Die Industrie merkte, dass es so nicht weitergehen konnte und entwarf erste Ans\"�tze zum Software Engineering.
\subsubsection{Definition}
Das Institute of Electrical and Electronics Engineers IEEE definiert in ihrem Standard IEEE 610.12 ''Software Engineering'' wie folgt:
\begin{quote}
''Die Anwendung eines systematischen, disziplinierten und quantifizierbaren Ansatzes auf die Entwicklung, den Betrieb und die Wartung von Software, das heisst, die Anwendung der Prinzipien des Ingenieurwesens auf Software.''
\end{quote}

\noindent Bruegge gibt einen Rahmen, in welchem Software Engineering zum Tragen kommt:
\begin{quote}
A collection of techniques, methodologies, and tools that help with the production of
\begin{itemize}
\item high quality software systems - with a given budget
\item before a given deadline
\item while change occurs
\end{itemize}
\end{quote}

\subsubsection{Rollen}
In einem Softwareentwicklungsprojekt kann es die folgenden Rollen geben:
\begin{itemize}
\item Architekt
\item Customer
\item Developer
\item Projekt Manager
\item Quality Manager
\item Requirements Engineer
\item Sales Manager
\item User Experience Designer
\end{itemize}

\subsubsection{T�tigkeiten}
% todo



\newpage

\section{Woche 2}

\newpage

\section{Woche 3}

\newpage

\section{Woche 4}

\newpage

\section{Woche 5}

\newpage

\section{Woche 6}

\newpage

\section{Woche 7}

\newpage

\section{Woche 8}

\newpage

\section{Woche 9}

\newpage

\section{Woche 10}

\newpage

\section{Woche 11}

\newpage

\section{Woche 12}

\newpage

\section{Woche 13}

\newpage

\section{Woche 14}

\newpage

\section{Woche 15}
\end{document}